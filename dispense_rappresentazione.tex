\documentclass[11pt]{article}


\usepackage{etex}
\usepackage[T1]{fontenc}
\usepackage[utf8]{inputenc}
\usepackage[italian]{babel}
\usepackage[a4paper]{geometry}
\usepackage[pdftex]{graphicx}

\usepackage{amsmath}
\usepackage{amssymb}
\usepackage{amsthm}
\usepackage{booktabs}
\usepackage{paralist}
\usepackage{subfig}
\usepackage{array}
\usepackage{xy}
\usepackage{multicol}
%\usepackage{slashbox}
\usepackage{fancyhdr}
\usepackage{makeidx}
\usepackage{hyperref}
\usepackage{wrapfig}
\usepackage[T1,OT1]{fontenc} 
\usepackage[nohug,small]{diagrams}


\usepackage{grffile}
\usepackage{tikz}
\usepackage{pgf,tikz}
\usetikzlibrary{shapes.geometric,calc}

\usetikzlibrary{arrows}
\topmargin 0cm
\oddsidemargin 0cm
\evensidemargin 0cm
\textwidth 16.5cm
\textheight	23.5cm
\marginparwidth 2cm
\marginparpush 2cm



\title{Dispense del corso di Teoria della Rappresentazione}
\author{Fabio Zoratti}
\date{\today}



\makeindex

\theoremstyle{plain}
\newtheorem{thm}{Teorema}[section]
\newtheorem{lemma}[thm]{Lemma}
\newtheorem{prop}[thm]{Proposizione}
\newtheorem{post}[thm]{Postulato}
\newtheorem*{cor}{Corollario}

\theoremstyle{definition}
\newtheorem{defn}{Definizione}[section]
\newtheorem{exmp}{Esempio}[section]
\newtheorem{prob}{Problema}[section]
\newtheorem{hint}{Suggerimento}[section]
\newtheorem{sol}{Soluzione}[section]
\newtheorem*{rem}{Osservazione}

\theoremstyle{remark}
\newtheorem*{note}{Nota}





\newcommand{\C}{\mathbb{C}}
\newcommand{\dsum}{\displaystyle\sum}
\newcommand{\dint}{\displaystyle\int}






















\begin{document}
\maketitle





\newpage
\section{Teoria dei gruppi}

\begin{defn}[Gruppo] Un gruppo è un insieme con associata un operazione binaria $\cdot : G\times G \to G$ che gode di alcune proprietà
\begin{enumerate}
	\item Associatività \quad $(ab)c = a(bc)$
	\item Esistenza unità \quad $ea = ae = a$
	\item Esistenza inverso $a'$ per ogni elemento $a$ \quad $a' a = a a' = e$
\end{enumerate}

\end{defn}

\paragraph{Esempi}
\begin{enumerate}
	\item $\mathbb{Z}, \mathbb{Q}, \mathbb{R}, \mathbb{C}$ con l'operazione di somma.
	\item $\mathbb{Q}^*, \mathbb{R}^*, \mathbb{C}^*$ con l'operazione di moltiplicazione. (Senza lo 0)
	\item $GL_n(\mathbb{R})$ oppure $GL(V)$
	\item $f:I\to I $ biunivoca, con $I$ insieme e con l'operazione di composizione. Nel caso in cui $I$ sia un insieme finito, tanto vale scegliere $I = \{1,2,3,\ldots, n\}$. In tal caso questo gruppo si chiama $S_n$
\end{enumerate}

\paragraph{Alcuni teoremi elementari}
\begin{enumerate}
	\item L'unità $e$ è unica
	
	Dimostrazione: supponiamo per assurdo che siano due distinte, $e, e'$. Allora vale
	
	\[e = ee' = e' \qed\]
	
	\item L'inverso è unico.

	Dimostrazione:
	
	Supponiamo per assurdo che siano due, $a', a''$
	
	\[(a' a)a'' = a'(aa'') \Rightarrow e a'' = a' e \qed \]
	
	\item Se ho $a_1, a_2, \ldots, a_n$, il prodotto di questi termini è ben definito senza bisogno di parentesi
	\item Esistono le potenze, ovvero $\forall k \in \mathbb{Z}, \forall a \in G \exists b\in G | a^k = b$
	
	Vale sempre la regola
	\[a^{k+h} = a^k \cdot a^h \]
	
	Ricorda che 
	
	\[ (ab)^{-1} = b^{-1}a^{-1}\]

\end{enumerate}



\begin{defn}[Sottogruppo]
Sia $G$ un gruppo, $H\subseteq G$ si dice sottogruppo di $G$ se:
\begin{itemize}
	\item $e\in H$
	\item $x,y\in H \Rightarrow xy\in H$
\end{itemize}
e si indica $H \leq G$.

\end{defn}

\begin{defn}[Sottogruppo normale]
Sia $G$ un gruppo, $H \leq G$ si dice \textit{normale} in $G$ se
\[
	\forall h\in H, \forall g\in G\qquad ghg^{-1}\in H
\]
e si indica $H \trianglelefteq G$.
\end{defn}

\begin{defn}[Gruppo quoziente]
NON LO SCRIVO PERCHÈ È LUNGO, ASPETTO DI VEDERE COME/SE LO DEFINISCE LUI
\end{defn}

\begin{defn}[Classi di coniugio]
Sia $G$ un gruppo, $x \in G$, la classe di coniugio di $x$ è l'insieme $\{ gxg^{-1} | g\in G \}$. Si dimostra facilmente che le classi di coniugio di tutti gli elementi di $G$ formano una partizione del gruppo stesso. Si osserva inoltre che un sottogruppo è normale se e solo se è unione di classi di coniugio (ATTENZIONE: è raro che unendo a caso classi di coniugio si ottenga un sottogruppo).
\end{defn}


\begin{exmp}[Le classi di coniugio di $GL_n(\C)$]
Nel caso del gruppo $GL_n(\C)$ due matrici stanno nella stessa classe di coniugio se e solo se sono simili, quindi per ogni classe di coniugio esiste un rappresentante canonico che è la forma di Jordan di una qualsiasi matrice nella classe (con opportune convenzioni sull'ordine dei blocchi e degli autovalori).
\end{exmp}

\begin{defn}[Centro di un gruppo]
	Sia $G$ un gruppo, il \textit{centro} di $G$ si indica con $Z(G)$ ed è il sottoinsieme degli elementi che commutano con tutto $G$:
	\[
		Z(G)=\{ h\in G\ |\ hg=gh\ \forall g\in G \}
	\]
	\'E immediato verificare che $Z(G)$ è un sottogruppo normale di $G$.

\end{defn}

\begin{defn}[Prodotto diretto di gruppi]
Siano $G$ e $H$ gruppi. Si definisce prodotto diretto di $G$ e $H$ il gruppo $G \times H = \{ (g, h) | g \in G, h \in H\}$ con l'operazione componente per componente.
\end{defn}


\begin{defn}[Omomorfismo (isomorfismo) di gruppi]
Siano $G$ ed $H$ gruppi, un'applicazione $\varphi:G\to H$ si dice \textit{omomorfismo di gruppi} se
\[
	\forall g_1,g_2\in G\qquad \varphi(g_1 g_2)=\varphi(g_1)\varphi(g_2)
\]
dove la prima moltiplicazione è fatta in $G$ mentre la seconda in $H$.
Se $\varphi$ è bigettiva, allora si dice \textit{isomorfismo}.
\end{defn}


\begin{defn}[Azione di un gruppo su un insieme] Sia $G$ un gruppo e $I$ un insieme. Definiamo un azione $a$ di $G$ su $I$ una funzione $a:G\times I \to I$ che rispetti la regola di composizione, ovvero che se $h,h\in G$ e $i \in I$, valga

\[ a(h,a(g,i)) = a(hg, i) \]

Normalmente si usa una notazione abbreviata in cui invece di scrivere $a(g,i)$ si scrive direttamente $g\cdot i$ o addirittura $gi$


\label{defn:azione}
\end{defn}


\begin{defn}[Azione transitiva]
Un'azione di un gruppo $G$ su un insieme $I\neq \emptyset$ si dice \textit{transitiva} se $\forall\ i,j\in I\ \exists s\in G$ t.c. $j=s\cdot i$.
\label{defn:azione transitiva}
\end{defn}

SAREBBE UTILE SCRIVERE UN COMANDO PER SCRIVERE ORB(X) SOLO CHE NON SO COME SI FA...
\begin{defn}[Orbita di un elemento]
Sia $G$ un gruppo che agisce sull'insieme $I$, dato $x\in I$ si chiama \textit{orbita} di $x$ in $G$ l'insieme $Orb_{G}(x)=\{ g\cdot x\ |\ g\in G \}$, se il gruppo utilizzato è chiaro si può scrivere semplicemente $Orb(x)$. Si osserva subito che un'azione è transitiva se e solo se induce una unica orbita.
\label{defn:orbita}
\end{defn}
\begin{rem}
	Le classi di coniugio sono le orbite degli elementi generate mediante l'azione per coniugio.
\end{rem}




\begin{defn}[Azione semplicemente transitiva]
Un'azione di $G$ su un insieme $I\neq \emptyset$ di dice \textit{semplicemente transitiva} se $\forall\ i,j\in I\ \exists !\ s\in G$ t.c. $j=s\cdot i$.
\end{defn}


\begin{defn}[Funzione $G$ equivariante]

Dato un gruppo $G$ che agisce su due insiemi $I$ e $J$, una funzione $\phi: I \to J$ si dice $G$ equivariante se 

\[ \phi(s \cdot_I i) = s \cdot_J \phi(i) \qquad \forall s \in G, \ \ \forall i \in I \]


\end{defn}


















\newpage
\subsection{Proprietà dei gruppi abeliani}



\begin{thm}Ogni gruppo abeliano finito è isomorfo al prodotto di gruppi ciclici.


\end{thm}

\begin{rem} Sia $G$ un gruppo abeliano. Allora 

\[ |G| = card(\{Hom(G \to \C^*) \})\]


Se invece $G$ non è abeliano allora nella formula precedente all'uguale va sostituito un $>$

\end{rem}


\subsection{Proprietà del gruppi simmetrici}

\begin{thm}[Ogni elemento $\sigma \in S_n$ si scrive in modo unico come prodotto di cicli disgiunti a meno dell'ordine dei fattori]


\end{thm}

\begin{prop}Il segno di un ciclo di lunghezza $k$ è esattamente $(-1)^{k-1}$


\end{prop}



\subsection{Proprietà dei gruppi ciclici}

\begin{rem}
Due gruppi ciclici dello stesso ordine sono isomorfi
\end{rem}

\subsection{Proprietà dei gruppi diedrali}

\begin{defn}[Gruppo diedrale]
L'insieme $D_n$ delle rotazioni e simmetrie di un poligono regolare di $n$ lati è un gruppo con l'operazione di composizione.
Detta $\rho$ una rotazione di $2\pi/n$ (che ha ordine $n$, e per inverso ha $\rho^{n}$) e $\sigma$ una qualunque riflessione (che ha ordine $2$), esse generano
il gruppo $D_n$, che si può quindi presentare nel seguente modo: $$D_n=<\rho,\sigma|\rho^n=\sigma^2=id,\ \sigma\rho\sigma=\rho^{-1}>$$
\end{defn}

\begin{rem}
 Le $n$ potenze distinte di $\rho$ sono tutte e sole le rotazioni di $D_n$, mentre gli elementi della forma $\sigma\rho^{i},\ i=0,1,..,n-1$ 
 sono tutte e sole le riflessioni. 
\end{rem}

\begin{rem}
 Si dimostra facilmente che la relazione $\sigma\rho\sigma=\rho^{-1}$ è verificata da qualsiasi rotazione $\rho$
 e qualsiasi riflessione $\sigma$.
\end{rem}







\newpage
\section{Algebra multilineare}
\subsection{Alcune generalizzazioni di algebra lineare}

\begin{defn}[Base di uno spazio vettoriale]
Sia $V$ uno spazio vettoriale e $I$ un insieme; una base di $V$ è una funzione $e: I \to V$ tale che 
$\forall v \in V,\  \exists!\  a: I \to \mathbb{C}$ a supporto finito per cui vale $v=\sum_{i\in I}a_i e_i$. La funzione $a$ 
valutata in $i$ prende il valore della $i$-esima coordinata del vettore $v$ nella base $e$. Questa definizione è compatibile con la 
definizione di base come insieme di vettori generatori linearmente indipendenti.
\end{defn}


\begin{lemma}
 Sia $e:I\to V$ una base di $V$ e $W$ uno spazio vettoriale. $f: I \to W$ una funzione. Allora $\exists!\  \phi: V \to W$ lineare tale che

\[\phi(e_i) = f_i \]

Inoltre $\phi$ è un isomorfismo $\Leftrightarrow$ $f$ è una base.
\end{lemma}


\subsection{Prodotto tensoriale}


% \[ \tridiag{V\times W}{ \otimes }{V\otimes W}{\phi}{Z}{f} \]
  


\begin{defn}[Prodotto tensoriale]
   Siano $V, W$ due $\mathbb{C}$-spazi vettoriali. Si dice prodotto tensore di $V$ e $W$, 
   e si indica come $V\otimes W$, uno spazio vettoriale con una funzione bilineare 
   $\otimes: V \times W \to V\otimes W$ tale che per ogni data funzione bilineare $h: V\times W \to  Z$,
   esiste unica $\phi: V\otimes W \to Z$ lineare per cui $\phi(v \otimes w)=h(v,w)$.
   Questa proprietà viene detta proprietà universale e la funzione $\otimes: V \times W \to V\otimes W$
   viene detta funzione universale.
%	\tridiag{V\times W}{ \otimes }{V \otimes W}{\phi}{Z}{f}



\label{defn:prodotto tensoriale}
\end{defn}


\begin{prop}
Se ho due prodotti tensoriali $V \otimes W$ e $V \overline{\otimes} W$, allora esiste un unico isomorfismo 
$\phi: V \otimes W \to V \overline{\otimes} W$ tale che

\[ \phi (v\otimes w) = v \overline{\otimes} w\]
\end{prop}


\begin{note}
\'E importante notare che non tutti gli elementi $z \in V \otimes W$ si scrivono come $z = v \otimes w$. In particolare, per fare un esempio concreto che mostra che questa cosa non funziona, prendiamo $W = V^*$. Vedremo fra poco che $V\otimes V^*$ è canonicamente isomorfo allo spazio delle applicazioni bilineari da $V$ in $\C$, che sappiamo scriverlo come matrici $n\times n$. Tuttavia se un elemento si scrive in termini di matrici come $z = v\otimes w$, allora la matrice associata a $z$ in una base avrà rango al massimo 1, ben lontano da coprire tutto lo spazio.
\end{note}


\begin{prop}
\[\langle\{ v \otimes w | v \in V, w \in W\} \rangle  = V \otimes W\]

\end{prop}


\begin{defn}[Prodotto tensoriale di mappe lineari]

\end{defn}

\begin{rem}

\[ id_V \otimes id_W = id_{V\otimes W}\]
\end{rem}




\begin{prop}

Se $e_i$ è una base di $V$ e $f_i$ è una base di $W$ allora $e_i \otimes f_j$ è una base di $V \otimes W$
\end{prop}


\begin{cor}
\[dim(V \otimes W) = dim V \cdot dim W \]

\end{cor}

















\begin{defn}
DEFINISCI TRACCIA DEL PRODOTTO TENSORE, OVVERO 

\[ tr(f\otimes g)\]

\end{defn}


\begin{thm}
Se $f:V\to V$ e $g:W\to W$ sono endomorfismi di spazi vettoriali, allora vale la formula

\[tr(f\otimes g) = tr(f) tr(g)  \]

\end{thm}

\textsc{Dimostrazione:}


\subsection{Prodotto esterno e prodotto simmetrico}

\begin{defn}[Applicazione $r$-lineare simmetrica/alternante]
 Una applicazione $\phi: V^n \to Z$ si dice $r$-lineare se è lineare in ogni componente dopo aver fissato le altre $n-1$.

 Inoltre $\phi$ si dice simmetrica se $\phi(v_{s(1)},..,v_{s(n)})=\phi(v_1,..,v_n),\ \forall s \in S_n$, mentre si dice 
 alternante se $\phi(v_{s(1)},..,v_{s(n)})=\mathrm{sgn}(s)\phi(v_1,..,v_n),\ \forall s \in S_n$.
 
\end{defn}

\begin{prop}
 Un'applicazione $h: V^n \to W$ è alternante se e solo se $h(v_1,..,v_n)=0$ se $v_i=v_j$ per qualche $i\neq j$.
 \end{prop}

 \begin{prop}
  Un'applicazione $h: V^n \to W$ è nulla se i vettori $v_1,..,v_n$ sono linearmente dipendenti.
 \end{prop}

\begin{defn}[Prodotto esterno]
Sia $n$ un intero positivo, $V$ uno spazio vettoriale. Un prodotto esterno è uno spazio vettoriale indicato con $\bigwedge^n V$
dotato di una funzione $n$-lineare alternante $\wedge: V^n \to \bigwedge^n V$ che manda $(v_1,..,v_n)$ in 
$v_i\wedge v_2\wedge..\wedge v_n \in \bigwedge^n V$, tale che $\forall h: V^n \to Z$ $n$-lineare alternante, 
esiste unica $\phi: \bigwedge^n V \to Z $ lineare per cui vale $\phi(v_1\wedge v_2\wedge .. \wedge v_n)=h(v_1,..,v_n)$.

\label{defn:prodotto esterno}
\end{defn}





\begin{thm}[Dimensione del prodotto esterno]
Sia $V$ uno spazio vettoriale di dimensione $n$, $\{e_i| 1 \leq i \leq n\}$ una base di $V$ e $k$ un intero positivo.
Allora l'insieme $E=\{e_{i_1} \wedge e_{i_2} \wedge e_{i_k}| 1 \leq i_1 < i_2 <...< i_k \leq n\}$ è una base di $\bigwedge^k V$ 
e si ha $|E|= \binom {n}{k}$.

\label{thm:prodotto esterno}
\end{thm}


MANCANO UN SACCO DI PROPRIETA' E LE DIMOSTRAZIONI





\begin{defn}[Prodotto simmetrico]

Sia $n$ un intero positivo, $V$ uno spazio vettoriale. Un prodotto simmetrico è uno spazio vettoriale indicato con $S^n V$
dotato di una funzione $n$-lineare simmetrica $V^n \to \bigwedge^n V$ che manda $(v_1,..,v_n)$ in 
$v_i v_2..v_n \in S^n V$, tale che $\forall h: V^n \to Z$ $n$-lineare simmetrica, 
esiste unica $\phi: S^n V \to Z $ lineare per cui vale $\phi(v_1 v_2 .. v_n)=h(v_1,..,v_n)$.

\label{defn:prodotto simmetrico}
\end{defn}





\begin{thm}[Dimensione del prodotto simmetrico]

Sia $V$ uno spazio vettoriale di dimensione $n$, $\{e_i| 1 \leq i \leq n\}$ una base di $V$ e $k$ un intero positivo.
Allora l'insieme $E=\{e_{i_1} \wedge e_{i_2} \wedge e_{i_k}| 1 \leq i_1 \leq i_2 \leq...\leq i_k \leq n\}$ è una base di $S^k V$ 
e si ha $|E|= \binom {n+k-1}{k}$.

\label{thm:prodotto simmetrico}
\end{thm}




















\newpage
\section{Prime proprietà delle rappresentazioni}

\begin{defn}[Rappresentazione] Sia $G$ un gruppo. Una rappresentazione $\rho$ di $G$ è una coppia composta da uno spazio vettoriale di dimensione qualsiasi $V_\rho$ e una funzione $\rho: G \to GL(V_\rho)$ che manda ciascun elemento del gruppo in un'applicazione lineare di $V_\rho$, ovvero un suo endomorfismo. Affinché $\rho$ sia una rappresentazione deve essere un omomorfismo di gruppi, ovvero in parole semplici deve rispettare la regola di composizione. In formule, se $s, t \in G$ deve valere

\[ \rho(st) v = \rho(s)\rho(t) v \qquad \forall v \in V_\rho, \quad \forall s,t \in G\]

La dimensione di $V_\rho$ viene detta grado della rappresentazione.

\end{defn}

\begin{prop} $\rho(G)$ è evidentemente un sottogruppo di $GL(V_\rho)$, quindi esistono sempre inversi, potenze e tutte le cose che valgono per i gruppi.

\end{prop}


\textbf{Esempi.}
\begin{enumerate}
	\item La rappresentazione banale, di grado qualsiasi, indicata con $\rho_1$ che manda qualsiasi elemento di $g$ nell'identità di $V_\rho$, ovvero
	
	\[ \rho(s ) = id_{V_\rho} \qquad \forall s \in G\]
	\item Dato $S_n$, il segno di un elemento $s\in S_n$ è una rappresentazione di grado 1. Infatti si ha $sgn(st) = sgn(s) sgn(t)$.
	\item L'azione naturale di $S_n$ sui vettori della base. Prendiamo quindi $G = S_n$ e uno spazio vettoriale di dimensione $n$, che sarà sicuramente isomorfo a $\C^n$. Prendiamo la base canonica di $\C^n$ e la chiamiamo $e_i$. Descriviamo la rappresentazione $\rho: S_n \to GL(\C^n)$ dicendo cosa fa agli elementi della base. Per linearità si estenderà a tutto lo spazio.
	
	\[ \rho(s) e_i = e_{s(i)}\]
	
	Notare che in questo caso $deg(\rho) = n$. Notiamo inoltre che se rappresentiamo nella base canonica le matrici associate a $\rho(s)$ queste matrici sono unitarie. Inoltre, ogni colonna (e anche ogni riga) contiene esattamente un 1 e tutti gli altri sono 0.
	
	Prendiamo come esempio $S_3$ e vediamo cosa succede. Notiamo innanzitutto che $ |S_3| = 3! = 6$
	FINISCI DI SCRIVERE
\end{enumerate}








\begin{prop}
Sia $G$ un gruppo finito e $\rho: G \to GL(V_\rho)$ una sua rappresentazione. Allora $\forall g \in G$ la matrice $\rho(g)$ ammette una base di autovettori in $V_\rho$, ovvero è diagonalizzabile. Inoltre, tutti gli autovalori di $\rho(g)$ sono radici $n-$esime dell'unità.

\textsc{Nota bene:} Per ogni matrice in generale la base è diversa, quindi le varie matrici in generale \textbf{non} sono simultaneamente diagonalizzabili. In particolare, tutte le matrici $\rho(s)$ sono simultaneamente diagonalizzabili $\Leftrightarrow$ $G$ è abeliano. 

\label{prop:diagonalizzabilita rappresentazioni}
\end{prop}

\textsc{Dimostrazione:} Se $G$ è un gruppo finito, allora $\exists k | g^k = e$\footnote{Dato che $g$ è finito, se prendo l'insieme delle potenze $I = \{g^k| k\in \mathbb{Z}\}$, proprio perchè $G$ è finito si ha che $I$ ha un numero finito di elementi, quindi ci saranno $m,n \in \mathbb{Z}$ tali che $g^m = g^n = h$. Dato che nei gruppi esiste l'inverso, sarà $g^{n-m} = e$}. Dato che $\rho:G\to GL(V_\rho)$ mantiene queste proprietà in quanto omomorfismo, dovrà essere

\[ \rho(g)^k = id\]

Con il polinomio minimo si mostra facilmente che $\rho(g)$ è diagonalizzabile. MATEMATICI SCRIVETE IL PERCH\'E. Inoltre da questa formula è anche evidente che tutti gli autovalori di $\rho(g)$ hanno modulo $1$ e in particolare saranno radici $k-$esime dell'unità.

Ricordiamo un teorema di algebra lineare per finire l'ultima parte della dimostrazione: due endomorfismi di uno spazio vettoriale diagonalizzabili sono simultaneamente diagonalizzabili $\Leftrightarrow$ commutano. Da questo teorema segue facilmente la seconda parte dell'enunciato. \qed





\begin{defn}[Omomorfismo di rappresentazioni]
Siano $\rho$ e $\sigma$ due rappresentazioni di $G$ su $V_{\rho}$ e $V_{\sigma}$ rispettivamente, un omomorfismo di spazi vettorali $\varphi:V_{\rho}\to V_{\sigma}$ si dice \textit{omomorfismo di rappresentazioni} se
\[
	\forall\ a\in G, \forall\ v\in V_{\rho}\quad \varphi(\rho(a)(v)) = \sigma(a)(\varphi(v))
\]
oppure equivalentemente
\[
	\forall\ a\in G\quad \varphi\circ \rho(a) = \sigma(a)\circ \varphi
\]


%Siano $G, H$ due gruppi e $\rho: G \to V_\rho$ e $\sigma: H \to V_\sigma$ due loro rappresentazioni. Una funzione lineare da $V_\rho \to V_\sigma$ \footnote{Ovvero un omomorfismo da $V_\rho$ a $V_\sigma$} è un omomorfismo di rappresentazioni se rispetta la regola di composizione


%\[ \qquad \forall v,w \in V_\rho, V_\sigma\]


\end{defn}



\begin{defn}[Rappresentazioni isomorfe]
Due rappresentazioni si dicono \textit{isomorfe} se esiste un omomorfismo di rappresentazioni tra di loro che è anche bigettivo.
\end{defn}




\paragraph{Rappresentazioni di grado 1}

\begin{thm}[Le classi di isomorfismo delle rappresentazioni di grado 1 sono gli omomorfismi da $G$ in $\C^*$]


\end{thm}


\begin{exmp}[Rappresentazioni di grado 1 di $C_n$]

\end{exmp}


\begin{exmp}[Rappresentazioni di grado 1 di $S_3$]

\end{exmp}

\begin{exmp}[Rappresentazioni di grado 1 di $C_n \times C_n$]


(generalizzazione a prodotto di $C_{n_i}$)
\end{exmp}














\newpage
\subsection{Operazioni con le rappresentazioni}

\begin{defn}[Somma di rappresentazioni]

\label{defn:somma di rappresentazioni}
\end{defn}

Osservazioni:

\begin{enumerate}
\item $\rho + \sigma \cong \sigma + \rho$
\item $\rho + (\sigma + \tau) \cong (\rho + \sigma ) + \tau$
\item Esiste l'elemento neutro che è la rappresentazione di grado 0 ma non esiste l'inverso.

\end{enumerate}





\begin{defn}[Prodotto di rappresentazioni]

\label{defn:prodotto di rappresentazioni}
\end{defn}


Osservazioni:


\begin{enumerate}
\item $1\otimes \rho \cong \rho$
\item $\rho \otimes \sigma \cong \sigma \otimes \rho$
\item $0 \otimes \rho \cong 0$
\item $\rho \otimes (\sigma \otimes \tau) \cong (\rho \otimes \sigma)\otimes \tau$
\item $\rho \otimes (\sigma_1 + \sigma_2) \cong \rho \otimes \sigma_1 + \rho \otimes \sigma_2$

\end{enumerate}





\begin{defn}[Rappresentazione duale]

\label{defn:rappresentazione duale}
\end{defn}

Osservazione: vale

\[ (\rho + \sigma)^* \cong \rho^* + \sigma^* \]

E l'isomorfismo è canonico. SCRIVI DIMOSTRAZIONE.





\begin{defn}[Rappresentazione regolare]

\label{defn:rappresentazione regolare}
\end{defn}

\begin{exmp}[La rappresentazione regolare di $S_3$]


\end{exmp}


\begin{thm}

\[\mathcal{R}_G \cong \dsum_i deg(\rho_i) \rho_i \]

\end{thm}



\subsection{Sottospazi invarianti e scomposizione delle rappresentazioni}


\begin{defn}[Sottospazio invariante]
NON VA MESSO TUTTO ASSIEME NELLA DEFINIZIONE DI SOTTORAPPRESENTAZIONE???

\end{defn}

\begin{defn}[Sottorappresentazione]
Sia $\rho$ una rappresentazione di $G$ su $V_{\rho}$, una sottorappresentazione di $\rho$ è un sottospazio vettoriale $W\subseteq V_{\rho}$ tale che $\rho(s)(W)\subseteq W\ \forall\ s\in G$. Posso definire una rappresentazione $\sigma$ con $V_{\sigma}=W$ e $\sigma(s)=\rho(s)\restriction W$ (la indicherò con $\sigma\subseteq \rho$).
COME SI SCRIVE LA RESTRIZIONE DI UNA FUNZIONE A UN DOMINIO PIÙ PICCOLO?????

\end{defn}



\begin{defn}[Rappresentazione irriducibile]
Una rappresentazione $\rho$ di $G$ è \textit{irriducibile} se
\begin{enumerate}
	\item $\rho \neq 0$ ($deg(\rho) \geq 1$)
	\item $\rho$ non ha sottorappresentazioni non banali (diverse da 0 e $V_{\rho}$).
\end{enumerate}

\end{defn}

\begin{rem} Normalmente la cosa che si fa più spesso in teoria della rappresentazione è cercare di scomporre la rappresentazione di un gruppo come somma di rappresentazioni irriducibili. Vedremo quindi adesso diversi teoremi che ci aiuteranno in questi problemi.

\end{rem}



\begin{exmp}[Rappresentazione regolare di $S_3$]


\end{exmp}



\begin{thm}[Le rappresentazioni di un gruppo finito sono completamente riducibili]

\end{thm}

\begin{prop}[Prodotto hermitiano invariante]

\end{prop}


\begin{lemma}
Sia $h: V_\rho \times V_\rho \to \C$ una forma hermitiana definita positiva e invariante per $\rho: G \to GL(V_\rho)$ e sia $\rho|_W: G \to GL(W)$ una sottorappresentazione di $\rho$. Allora se $W^\perp$ è l'ortogonale di $W$, $\rho|_{W^\perp}: G \to GL(W^\perp)$ è una sottorappresentazione.




\end{lemma}






\begin{lemma}
Sia $\rho: G \to GL(V_\rho)$ una rappresentazione di un gruppo finito $G$. Sia $\rho|_W: G \to GL(W)$ una sottorappresentazione di $\rho$. Allora esiste una sottorappresentazione $\sigma: G \to GL(W')$ tale che

\[\rho = \rho|_W + \sigma \]
\end{lemma}






\begin{rem} Notare che il teorema precedente è falso per gruppi finiti. (Esempio con $\mathbb{Z}^+$ che Salvatore non ha scritto con cura. Porco salvatore)



\end{rem}



\begin{thm} Siano $\rho: G \to GL(V_\rho)$ e $\sigma: G \to GL(V_\sigma)$ sono rappresentazioni di $G$ e $f: V_\rho \to V_\sigma$ è un omomorfismo di rappresentazioni, allora $Im(f)$ è una sottorappresentazione di $\sigma$ e $Ker(f)$ è una sottorappresentazione di $V_\rho$

\end{thm}





\begin{thm}Sia $G$ un gruppo abeliano finito. Allora ogni rappresentazione di $G$ è isomorfa alla somma di rappresentazioni di grado 1.



\end{thm}


\begin{prop} La rappresentazione regolare $\mathcal{R}$ di $C_n$ è isomorfa alla somma delle $n$ rappresentazioni irriducibili di grado 1 di $C_n$.

\end{prop}


\begin{lemma}
Date $\rho_1, \rho_2, \sigma$ rappresentazioni di $G$, allora

\[Hom(\rho_1 + \rho_2, \sigma) \cong Hom(\rho_1, \sigma) \oplus Hom(\rho_2, \sigma)\]

\end{lemma}


\begin{thm}[Lemma di Schur]


\end{thm}



\begin{thm}
Sia $\rho: G \to GL(V_\rho)$ una rappresentazione e 

\[\rho = \dsum_{i=1}^N n_i \rho_i \]

una sua scomposizione come somma di rappresentazioni irriducibili a due a due non isomorfe. Allora la scomposizione è unica.
\end{thm}



\begin{lemma}
Sia $\rho$ una rappresentazione di $G$ e $\mathcal{R}$ la sua rappresentazione regolare. Allora 
\[deg(\rho) = dim(Hom(\mathcal{R}, \rho))\]
\end{lemma}




\begin{thm}
Sia $\mathcal{R}$ la rappresentazione regolare di $G$, un gruppo finito, e sia 

\[ \mathcal{R} = \dsum_{i=1}^Nn_i \rho_i\]

Con $\rho_i$ irriducibili e a due a due non isomorfe. Allora ogni rappresentazione irriducibile di $G$ è isomorfa ad una $\rho_i$. Inoltre $n_i = deg(\rho_i)$ 
\label{thm: teorema importantissimo}
\end{thm}



\begin{cor}
Se $G$ è abeliano allora ha $|G|$ rappresentazioni irriducibili di grado 1 e $\mathcal{R}$ è la somma di queste.
\end{cor}



\begin{cor}
Sia $G$ un gruppo finito. $G$ ha un numero finito di rappresentazioni irriducibili, a meno di isomorfismi. Inoltre
\[|G| = \dsum n_i^2\]
\end{cor}



















\newpage
\section{Teoria dei caratteri}


\begin{defn}
Sia $\rho: G \to GL(V_\rho)$ una rappresentazione di un gruppo $G$. Definiamo carattere di $\rho$ la funzione che associa ad ogni elemento del gruppo $G$ la traccia della matrice associata all'elemento, ovvero

\[\chi_\rho(s) := tr (\rho(s)) \qquad \forall s \in G \]
Notare che $\chi_{\rho}$ è una funzione che va dal gruppo in $\C$, ovvero $\chi_{\rho}: G \to \C$

\end{defn}

Vediamo delle proprietà elementari del carattere

\textsc{Osservazioni:}
\begin{enumerate}
	\item Se $deg(\rho) = 1$ allora il carattere di $s$ è uguale a $\rho(s)$
	\item $\chi_{\rho_1} = deg(\rho)$. \footnote{Al solito $\rho_1$ è la rappresentazione che manda ogni elemento nell'identità di $V_\rho$}\\
	Questo è vero poichè $[\rho_1]=I_n\Rightarrow tr(\rho_1)=n$ ed $n=deg(\rho)$.
	\item $\chi_{\rho + \sigma}(s) = \chi_\rho(s) + \chi_\sigma(s)$.\\ 
	Questo è dovuto al fatto che la somma di rappresentazioni si può scrivere come matrice a blocchi. Una volta scritto così è evidente il risultato.
	\item $\chi_{\rho\sigma}(s) = \chi_\rho(s)\chi_\sigma(s)$.\\ 
	Questo deriva dal seguente fatto generale:
	
\begin{lemma} 
Se $f: V \to V$ e $g: W \to W$ sono endomorfismi di spazi vettoriali, allora $tr(f \otimes g) = tr(f)tr(g)$.
\end{lemma}
\textbf{Dimostrazione:} Iniziamo a considerare il caso in cui sia $f$ che $g$ siano diagonalizzabili: prendendo due basi $a:I\rightarrow V$ , $b:J\rightarrow W$ di autovettori rispettivamente per $f$ e per $g$, si verifica facilmente la verità della proposizione nella base indotta su $V\otimes W$ (ovvero in quella formata dagli $a_i\otimes b_j$).\\
Ora, essendo la traccia una funzione continua e le matrice diagonalizzabili dense nello spazio delle matrici, la proprietà affermata dal lemma si estende al caso generale per continuità.
	\item $\chi_{\rho}(s^{-1})=\overline{\chi_{\rho}(s)}$\\
Essendo $G$ un gruppo finito, $\forall s\in G\ \rho(s)^n = id$ dove $n=|G|$: dunque tutti gli autovalori di $\rho(s)$ sono radici ennesime dell'unità e $\rho(s)$ è diagonalizzabile\footnote{Si veda la proposizione \ref{prop:diagonalizzabilita rappresentazioni}}. In tale base è evidente che:
$$\chi_{\rho}(s^{-1})=tr(\rho (s^{-1}))=tr(\rho (s)^{-1})=\sum_i\lambda_i^{-1}=\sum_i\overline{\lambda_i}=\overline{tr(\rho(s))}=\overline{\chi_{\rho}(s)}$$
in quanto, avendo gli autovalori modulo 1, l'inverso coincide con il coniugio.  	
	\item $\chi_{\rho^*}(s)\footnote{Ricordiamo che $\rho^*(s) = (\rho(s)^{-1})^*$} = \overline{\chi_\rho(s)}$.\\
		Per l'osservazione precedente vale che
		$$\chi_{\rho^*}(s)=tr(^t\rho(s^{-1}))=tr(\rho(s^{-1}))=\overline{tr(\rho(s))}=\overline{\chi_\rho(s)}$$
	\item $\chi_{\rho}(hsh^{-1})=\chi_{\rho}(s)$ ovvero $\chi_\rho$ è costante sulle classi di coniugio di $G$. La motivazione è semplice: se due elementi sono coniugati tra loro questo significa che le matrici corrispondenti saranno simili e la traccia è un invariante di similitudine.
	
Di conseguenza, non sarà necessario calcolare il carattere per ogni elemento del gruppo ma basterà farlo per le classi di coniugio di $G$.

Le funzioni che costanti sulle classi di coniugio di un gruppo vengono dette $funzioni\ di\ classe$.
	\item Supponiamo di avere una rappresentazione per permutazioni. Sia $I$ un insieme finito e $G$ un gruppo allora 
$$\chi_{\rho_{I}}(s)=\#punti\ fissi\ di\ \rho_I(s)=|I^s|$$
dove $I^s:=\{i\in I| s\circ i=i\}$. La veridicità di questo fatto si vede scrivendo esplicitamente la matrice che rappresenta $\rho_I(s)$.
\end{enumerate}
\textbf{Esempio:} $G=S_3$, $I=\{1,2,3\}$. Allora

\[ \chi_{\rho_I}(s) = \begin{cases}
3 \qquad \text{se } s=id \\
1 \qquad \text{se } s\ \text{è una trasposizione}\\
0 \qquad \text{se } s\ \text{è un treciclo}\\
\end{cases} \]
Ricordandoci che $\chi_{\rho_I}=\chi_{1+\rho}$ si ha che 
\[ \chi_{\rho}(s) = \begin{cases}
2 \qquad \ \ \text{se } s=id \\
0 \qquad \ \ \text{se } s\ \text{è una trasposizione}\\
-1\qquad \text{se } s\ \text{è un treciclo}\\
\end{cases} \]

\begin{defn}[Prodotto hermitiano dei caratteri]

\[ \langle f | g \rangle = \dfrac{1}{|G|} \dsum_{s \in G} \overline{f(s)} g(s) \]

\end{defn}


\begin{thm}[Relazioni di ortogonalità]
Se $\rho$ e $\sigma$ sono rappresentazioni irriducibili di $G$, allora vale

\[\langle \chi_{\rho}|\chi_{\sigma} \rangle = \begin{cases}
1 \qquad \text{se } \rho \cong \sigma \\
0 \qquad \text{altrimenti }\\
\end{cases} \]
\label{relazione di ortogonalita}
\end{thm}

Per dimostrare questo teorema abbiamo bisogno di un lemma che ora enunciamo e dimostriamo.




\begin{lemma}
Se $(\rho, V_\rho)$ e $(\sigma, V_\sigma)$ sono rappresentazioni \footnote{Non necessariamente irriducibili} di $G$, allora vale

\[ \langle \chi_\rho | \chi_\sigma \rangle  = dimHom (\sigma, \rho)\] 

\end{lemma}
\textsc{Dimostrazione:}

L'idea principale per dimostrare questo lemma è di ridurci al caso più facile in cui una delle due rappresentazioni è quella banale. Per farlo notiamo un paio di cose

\[ \langle \chi_\rho | \chi_\sigma \rangle = \dfrac{1}{|G|} \dsum_{s\in G} \overline{\chi_\rho(s)} \chi_\sigma(s) = \dfrac{1}{|G|} \dsum_{s\in G} {\chi_{\rho^*}(s)} \chi_\sigma(s) = \dfrac{1}{|G|} \dsum_{s\in G} \chi_{\rho^*\sigma}(s)  = \langle 1 | \chi_{\rho^* \sigma} \rangle \]

Siamo passati da due rappresentazioni ad una sola. In particolare lo spazio vettoriale su cui agisce questa rappresentazione è 

\[ V_{\rho^* \sigma} = V_{\rho}^* \otimes V_\sigma \cong Hom(V_\rho, V_\sigma)\]

E questo isomorfismo segue semplicemente dalle proprietà del prodotto tensore di spazi vettoriali. Notiamo che sullo spazio degli omomorfismi\footnote{Dato che sono spazi vettoriali in questo caso si tratta semplicemente di applicazioni lineari} $Z = Hom(V_\rho, V_\sigma)$ è possibile definire una rappresentazione completamente analoga a $\rho^s\sigma$ in questo modo: se $f \in Z$, allora possiamo definire la rappresentazione $\tau, V_\tau = Z$ di $G$ in questo modo

\[ \tau(s)f = \rho(s) \circ f \circ \sigma^{-1}(s)\]

\'E possibile mostrare VI PREGO QUALCHE MATEMATICO LO FACCIA che se chiamo $\phi$ la mappa tale che

\[
\begin{cases}
V_\rho^* \otimes V_\sigma \xrightarrow{\phi} Hom(V_\rho, V_\sigma)\\
\rho^*\sigma \xrightarrow{\phi} \tau \\
\end{cases}
\]

Allora $\phi$ è un isomorfismo di rappresentazioni. A questo punto possiamo andare a cercare i sottospazi invarianti per $\tau$, ovvero stiamo andando a cercare le sottorappresentazioni irriducibili di $\tau$ sperando di usare teoremi che già conosciamo. In particolare stiamo quindi cercando dei sottospazi $W \subset Z = Hom(V_\rho, V_\sigma)$ tali che $\tau(s) W \subset W \quad \forall s \in G$ 

LA DIMOSTRAZIONE VA CONCLUSA QUANDO IL PROF FINISCE DI DIMOSTRARLA MARTED\'I

\qed



\textsc{Dimostrazione del teorema \ref{relazione di ortogonalita}:}

A questo punto la tesi del teorema \ref{relazione di ortogonalita} segue dal lemma precedente applicato insieme al lemma di Schur. \qed


\textsc{Osservazioni:}

\begin{itemize}
\item Ricordiamo che se $\rho$ è una rappresentazione di $G$, allora $\rho$ si può scrivere in modo unico come 

\[ \rho = \dsum_i n_i \rho_i\]

Dove le $\rho_i$ sono le rappresentazioni irriducibili di $G$ e gli $n_i$ sono numeri naturali $\geq 0$. Dall'equazione scritta sopra segue subito che

\[ \chi_\rho = \dsum_i n_i \chi_{\rho_i}\]

E possiamo ottenere un'informazione utile prendendo il prodotto scalare dell'equazione precedente con il carattere di una delle rappresentazioni $\rho_i$

\[ \langle \chi_\rho | \chi_{\rho_j} \rangle = \dsum_i n_i \langle \chi_{\rho_i} | \chi_{\rho_j} \rangle \Rightarrow n_i \delta_{ij} = \langle \chi_\rho | \chi_{\rho_j} \rangle \Rightarrow n_i = \langle \chi_\rho | \chi_{\rho_i} \rangle\]




\item Se $\rho$ e $\sigma$ sono 2 rappresentazioni irriducibili allora $$\rho \cong \sigma \Leftrightarrow \chi_{\rho}=\chi_{\sigma}$$

\item $\langle \chi_\rho | \chi_\rho \rangle = |\chi_\rho|^2 = \sum_i n_i^2$.
\item Conseguenza dell'ultima osservazione è che una rappresentazione di un gruppo $\rho$ è irriducibile $\Leftrightarrow \langle \chi_\rho | \chi_\rho \rangle = |\chi_\rho|^2 = 1$ 










\end{itemize}







\subsection{Esempi di rappresentazioni di gruppi finiti}

\begin{exmp}[Tabella dei caratteri di $S_3$]

La prima cosa da fare per costruire la tabella dei caratteri è vedere quanti elementi ha $S_3$, suddividerli in classi di coniugio e poi cercare le rappresentazioni irriducibili solo dopo aver fatto tutto questo. Notiamo subito che $S_3$ ha esattamente 3 classi di coniugio. La prima è ovviamente quella banale, composta solo dall'identità $e$. Poi c'è la classe delle trasposizioni $\{(1 2) ,(2 3), (1 3)\}$ che ha 3 elementi e poi ci sono i $3$cicli, ovvero $(1 2 3)$ e $(1 3 2)$. Possiamo cominciare a scrivere una tabella vuota $3\times 3$




\begin{table}[!ht]
\centering
\begin{tabular}{|c|c|c|c|}
\hline
$S_3$  & $e$ & $(1 2)$ & (1 2 3 )    \\
 & 1 & 3 & 2 \\
\hline
 & &  & \\
\hline
& &  & \\
\hline
& &  & \\
\hline
\end{tabular}
\end{table}



Una rappresentazione irriducibile che c'è sempre è la rappresentazione banale di grado 1, ovvero quella che manda ogni elemento nell'identità. La tabella con questa informazione diventa



\begin{table}[!ht]
\centering
\begin{tabular}{|c|c|c|c|}
\hline
$S_3$  & $e$ & $(1 2)$ & (1 2 3 )    \\
 & 1 & 3 & 2 \\
\hline
 $\rho_1$ & 1 & 1  & 1 \\
\hline
& &  & \\
\hline
& &  & \\
\hline
\end{tabular}
\end{table}


Un'altra rappresentazione che già conosciamo è il segno, $\epsilon$, che ricordiamo vale $(-1)^{n-1}$ dove $n$ è la lunghezza del ciclo. La tabella diventa




\begin{table}[!ht]
\centering
\begin{tabular}{|c|c|c|c|}
\hline
$S_3$  & $e$ & $(1 2)$ & (1 2 3 )    \\
 & 1 & 3 & 2 \\
\hline
 $\rho_1$ & 1 & 1  & 1 \\
\hline
$\epsilon$ & 1 & -1 & 1 \\
\hline
& &  & \\
\hline
\end{tabular}
\end{table}

A questo punto ci sono due motivi per dire che l'ultima rappresentazione ha grado 2: il primo è che è l'unico modo di ottenere la relazione

\[ |G | = \dsum_i n_i^2 \]

Il secondo è che se fossero due rappresentazioni di grado 1 allora il gruppo avrebbe solo rappresentazioni irriducibili di grado 1 e un teorema che abbiamo fatto implicherebbe che $S_3$ sia abeliano, cosa palesemente falsa. 

Per trovare il carattere dell'ultima rappresentazione possiamo agire in più modi. Innanzitutto la tabella ora ha la forma




\begin{table}[!ht]
\centering
\begin{tabular}{|c|c|c|c|}
\hline
$S_3$  & $e$ & $(1 2)$ & (1 2 3 )    \\
 & 1 & 3 & 2 \\
\hline
 $\rho_1$ & 1 & 1  & 1 \\
\hline
$\epsilon$ & 1 & -1 & 1 \\
\hline
$\rho$ & 2 &  & \\
\hline
\end{tabular}
\end{table}


In generale ci saranno due numeri complessi $a, b$ nelle due caselle che mancano. Tuttavia noi sappiamo un sacco di teoremi che ci permettono di restringere il campo dei valori che possono avere. Per esempio noi sappiamo che 

\[\langle \rho_i | \rho_j \rangle = \delta_{ij}\]
 
Per cui imponendo che il prodotto scalare con entrambe le precedenti faccia 0 abbiamo due equazioni e due incognite, ovvero un problema risolvibile. L'altro modo è dire che

\[ \mathcal{R} = 1 + \epsilon + 2\rho\]

E dato che il carattere si comporta bene con la somma di rappresentazioni, 

\[\chi_{\mathcal{R}} = \chi_1 + \chi_\epsilon + 2 \chi_\rho  \]

Ma sappiamo anche che 

\[ \chi_{\mathcal{R}}(s) = 
\begin{cases}
|G| \quad \text{se } s = e \\
0 \quad \text{altrimenti}
\end{cases}\]

Per cui con agili conti riusciamo a completare la tabella








\begin{table}[!ht]
\centering
\begin{tabular}{|c|c|c|c|}
\hline
$S_3$  & $e$ & $(1 2)$ & (1 2 3 )    \\
 & 1 & 3 & 2 \\
\hline
 $\rho_1$ & 1 & 1  & 1 \\
\hline
$\epsilon$ & 1 & -1 & 1 \\
\hline
$\rho$ & 2 & 0 & 1 \\
\hline
\end{tabular}
\caption{Tabella dei caratteri di $S_3$}
\label{tabella caratteri s3}
\end{table}

L'ultimo modo è cercare di scomporre un'altra rappresentazione a caso di $S_3$, cercando di trovare la rappresentazione che ci manca. Per esempio ricordiamo l'azione di $S_3$ sui vettori di base di $\mathbb{R}^3$

\[ \tau(s) e_i = e_{s(i)}\]

Ricordiamo che il sottospazio di dimensione $1$ fatto dallo span del vettore $v = e_1 + e_2 + e_3$ è un sottospazio invariante in cui $\tau(s)$ è sostanzialmente l'identità. Il suo ortogonale è un altro sottospazio invariante su cui $\rho$ è irriducibile. Di conseguenza potremo scrivere

\[ \tau = 1 + \rho\]

E siamo sicuri che l'altra rappresentazione di grado 2 sia esattamente quella che stiamo cercando proprio grazie al teorema che ci dice che tutte le rappresentazioni irriducibili di un gruppo compaiono nella sua rappresentazione regolare. (Teorema \ref{thm: teorema importantissimo})

Dato che è facile calcolare il carattere di $\tau(s)$ in quanto è uguale a $Fix(s)$, possiamo scrivere

\[ Fix(s) = 1 + \chi_\rho\]

Da cui si ricava subito il carattere della rappresentazione $\rho$



\end{exmp}






\begin{exmp}[Tabella dei caratteri di $S_4$]
Facciamo la prima cosa importante: dividiamo $S_4$ in classi di coniugio. Per i soliti teoremi sugli $S_n$, le classi di coniugio saranno 
\[\{e\}, \{(a b)\}, \{(a b c)\}, \{(a b c d)\}, \{(a b)(c d)\}\]

E notiamo che sono 5. Possiamo quindi cominciare a compilare la tabella dei caratteri vuota



\begin{table}[!ht]
\centering
\begin{tabular}{|c|c|c|c|c|c|}
\hline
$S_4$  & $e$ & $(1 2)$ & (1 2 3 ) & $(1 2 3 4)$ & $(1 2)(3 4)$ \\
 & 1 & 6 & 8 & 6 & 3 \\
\hline
 $\rho_1$ & 1 & 1  & 1 & 1 & 1\\
\hline
& &  & & & \\
\hline
& &  & & & \\
\hline
& &  & & & \\
\hline
& &  & & & \\
\hline
\end{tabular}
\end{table}


Dove ho già messo la rappresentazione banale. Anche per $S_4$, essendo un gruppo simmetrico c'è la rappresentazione segno di grado 1. 




\begin{table}[!ht]
\centering
\begin{tabular}{|c|c|c|c|c|c|}
\hline
$S_4$  & $e$ & $(1 2)$ & (1 2 3 ) & $(1 2 3 4)$ & $(1 2)(3 4)$ \\
 & 1 & 6 & 8 & 6 & 3 \\
\hline
 $\rho_1$ & 1 & 1  & 1 & 1 & 1\\
\hline
$\epsilon$ & 1  & -1 & 1 & -1 & 1 \\
\hline
& &  & & & \\
\hline
& &  & & & \\
\hline
& &  & & & \\
\hline
\end{tabular}
\end{table}

Dato che $S_4$ ha $4! = 24$ elementi, dobbiamo adesso trovare un modo per ottenere 22 come somma di 3 quadrati. Si vede subito che devono essere $ <4 $, ma se fossero anche tutti al massimo 2 non ce la faremmo, per cui ne esiste almeno una di grado 3.

\[ 22 = a^2 + b^2 + 3^2 \Rightarrow a^2 + b^2 = 13 \]

A questo punto è facile vedere che l'unica soluzione è $(2,3)$, ovviamente a meno dell'ordine. La tabella diventa


\begin{table}[!ht]
\centering
\begin{tabular}{|c|c|c|c|c|c|}
\hline
$S_4$  & $e$ & $(1 2)$ & (1 2 3 ) & $(1 2 3 4)$ & $(1 2)(3 4)$ \\
 & 1 & 6 & 8 & 6 & 3 \\
\hline
 $\rho_1$ & 1 & 1  & 1 & 1 & 1\\
\hline
$\epsilon$ & 1  & -1 & 1 & -1 & 1 \\
\hline
& 2&  & & & \\
\hline
& 3&  & & & \\
\hline
& 3&  & & & \\
\hline
\end{tabular}
\end{table}
 


A questo punto bisogna fare cose a caso cercando le rappresentazioni irriducibili. Per esempio possiamo di nuovo considerare la rappresentazione per permutazioni



\[ \tau(s) e_i = e_{s(i)}\]


Che si scompone anche questa come

\[ \tau = 1 + \rho\]

Vorremmo sapere se $\rho$ è irriducibile. Potremmo invocare qualche teorema ma lo faremo con le mani calcolando il carattere di $\rho$


\[ \chi_\rho(s) = Fix(s) - 1 = 
\begin{cases}
3 \quad \text{Se } s = e \\
1 \quad \text{Se } s = (a b) \\
0 \quad \text{Se } s = (a b c) \\
-1 \quad \text{Se } s = (a b c d ), (a b) (c d)\\
\end{cases}
\]

E andando a calcolare

\[\langle\chi_\rho |\chi_\rho\rangle = \dfrac{1}{24}\left(3^2  + 6 \cdot 1^2  + 0 + (-1)^2 \cdot (3 +6 )\right) = 1\]
 

Per cui è effettivamente irriducibile.  Aggiungiamola alla tabella



\begin{table}[!ht]
\centering
\begin{tabular}{|c|c|c|c|c|c|}
\hline
$S_4$  & $e$ & $(1 2)$ & (1 2 3 ) & $(1 2 3 4)$ & $(1 2)(3 4)$ \\
 & 1 & 6 & 8 & 6 & 3 \\
\hline
 $\rho_1$ & 1 & 1  & 1 & 1 & 1\\
\hline
$\epsilon$ & 1  & -1 & 1 & -1 & 1 \\
\hline
& 2&  & & & \\
\hline
$\rho$& 3 & 1 & 0 & -1 & -1\\
\hline
& 3&  & & & \\
\hline
\end{tabular}
\end{table}


A questo punto possiamo considerare $\rho\epsilon$ come altra rappresentazione. DIMOSTRA PRIMA CHE \'E UNA RAPPRESENTAZIONE DI $|G|$ ED \'E IRRIDUCIBILE


\begin{table}[!ht]
\centering
\begin{tabular}{|c|c|c|c|c|c|}
\hline
$S_4$  & $e$ & $(1 2)$ & (1 2 3 ) & $(1 2 3 4)$ & $(1 2)(3 4)$ \\
 & 1 & 6 & 8 & 6 & 3 \\
\hline
 $\rho_1$ & 1 & 1  & 1 & 1 & 1\\
\hline
$\epsilon$ & 1  & -1 & 1 & -1 & 1 \\
\hline
& 2&  & & & \\
\hline
$\rho$& 3 & 1 & 0 & -1 & -1\\
\hline
$\rho\epsilon$& 3 & -1 & 0 & 1 & -1\\
\hline
\end{tabular}
\end{table}


E a questo punto dato che ce ne manca solo una possiamo usare il trucco di prima e concludere

\begin{table}[!ht]
\centering
\begin{tabular}{|c|c|c|c|c|c|}
\hline
$S_4$  & $e$ & $(1 2)$ & (1 2 3 ) & $(1 2 3 4)$ & $(1 2)(3 4)$ \\
 & 1 & 6 & 8 & 6 & 3 \\
\hline
 $\rho_1$ & 1 & 1  & 1 & 1 & 1\\
\hline
$\epsilon$ & 1  & -1 & 1 & -1 & 1 \\
\hline
 $\sigma$& 2&  0 & -1& 0 & 2\\
\hline
$\rho$& 3 & 1 & 0 & -1 & -1\\
\hline
$\rho\epsilon$& 3 & -1 & 0 & 1 & -1\\
\hline
\end{tabular}
\caption{Tabella dei caratteri di $S_4$}
\label{tabella caratteri s4}
\end{table}



\end{exmp}





\begin{exmp}[Tabella dei caratteri di $D_5$]

La prima cosa da fare è dividere $D_5$ in classi di coniugio


FINIRE





\subsubsection{I problemi della prima lezione visti con i nuovi strumenti}
\begin{exmp}[Problema 1 prima lezione]


\end{exmp}

\begin{exmp}[Problema 2 prima lezione]


\end{exmp}

\begin{exmp}[Problema 3 prima lezione]


\end{exmp}
















\end{exmp}


















\end{document}
