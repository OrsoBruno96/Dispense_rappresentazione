\documentclass[11pt]{article}


\usepackage{etex}
\usepackage[T1]{fontenc}
\usepackage[utf8]{inputenc}
\usepackage[italian]{babel}
\usepackage[a4paper]{geometry}
\usepackage[pdftex]{graphicx}

\usepackage{amsmath}
\usepackage{amssymb}
\usepackage{amsthm}
\usepackage{booktabs}
\usepackage{paralist}
\usepackage{subfig}
\usepackage{array}
\usepackage{xy}
\usepackage{multicol}
\usepackage{slashbox}
\usepackage{fancyhdr}
\usepackage{makeidx}
\usepackage{hyperref}
\usepackage{wrapfig}
\usepackage[T1,OT1]{fontenc} 
\usepackage{chemfig}
\usepackage{epigraph}


\usepackage{grffile}
\usepackage{tikz}
\usepackage{pgf,tikz}
\usetikzlibrary{shapes.geometric,calc}

\usetikzlibrary{arrows}
\topmargin 0cm
\oddsidemargin 0cm
\evensidemargin 0cm
\textwidth 16.5cm
\textheight	23.5cm
\marginparwidth 2cm
\marginparpush 2cm



\title{Dispense del corso di Teoria della Rappresentazione}
\author{Fabio Zoratti}
\date{\today}





\theoremstyle{plain}
\newtheorem{thm}{Teorema}[section]
\newtheorem{lem}[thm]{Lemma}
\newtheorem{prop}[thm]{Proposizione}
\newtheorem{post}[thm]{Postulato}
\newtheorem*{cor}{Corollario}

\theoremstyle{definition}
\newtheorem{defn}{Definizione}[section]
\newtheorem{exmp}{Esempio}[section]
\newtheorem{prob}{Problema}[section]
\newtheorem{hint}{Suggerimento}[section]
\newtheorem{sol}{Soluzione}[section]
\newtheorem*{rem}{Osservazione}

\theoremstyle{remark}
\newtheorem*{note}{Nota}





\newcommand{\C}{\mathbb{C}}
























\begin{document}
\maketitle

Aggiungi l'esempio della prima lezione


\section{Teoria dei gruppi}

\begin{defn}[Gruppo] Un gruppo è un insieme con associata un operazione binaria $\cdot : G\times G \to G$ che gode di alcune proprietà
\begin{enumerate}
	\item Associatività \quad $(ab)c = a(bc)$
	\item Esistenza unità \quad $ea = ae = a$
	\item Esistenza inverso $a'$ per ogni elemento $a$ \quad $a' a = a a' = e$
\end{enumerate}

\end{defn}

\paragraph{Esempi}
\begin{enumerate}
	\item $\mathbb{Z}, \mathbb{Q}, \mathbb{R}, \mathbb{C}$ con l'operazione di somma.
	\item $\mathbb{Q}^*, \mathbb{R}^*, \mathbb{C}^*$ con l'operazione di moltiplicazione. (Senza lo 0)
	\item $GL_n(\mathbb{R})$ oppure $GL(V)$
	\item $f:I\to I $ biunivoca, con $I$ insieme e con l'operazione di composizione. Nel caso in cui $I$ sia un insieme finito, tanto vale scegliere $I = \{1,2,3,\ldots, n\}$. In tal caso questo gruppo si chiama $S_n$
\end{enumerate}

\paragraph{Alcuni teoremi elementari}
\begin{enumerate}
	\item L'unità $e$ è unica
	
	Dimostrazione: supponiamo per assurdo che siano due distinte, $e, e'$. Allora vale
	
	\[e = ee' = e' \qed\]
	
	\item L'inverso è unico.

	Dimostrazione:
	
	Supponiamo per assurdo che siano due, $a', a''$
	
	\[(a' a)a'' = a'(aa'') \Rightarrow e a'' = a' e \qed \]
	
	\item Se ho $a_1, a_2, \ldots, a_n$, il prodotto di questi termini è ben definito senza bisogno di parentesi
	\item Esistono le potenze, ovvero $\forall k \in \mathbb{Z}, \forall a \in G \exists b\in G | a^k = b$
	
	Vale sempre la regola
	\[a^{k+h} = a^k \cdot a^h \]
	
	Ricorda che 
	
	\[ (ab)^{-1} = b^{-1}a^{-1}\]

\end{enumerate}










\begin{defn}[Azione di un gruppo su un insieme]

\label{defn:azione}
\end{defn}


\begin{defn}[Azione transitiva]



\label{defn:azione transitiva}
\end{defn}


\begin{defn}[Orbita di un elemento]


\label{defn:orbita}
\end{defn}








\newpage
\subsection{Un esempio ricorrente: $S_n$}

\begin{thm}[Ogni elemento $\sigma \in S_n$ si scrive in modo unico come prodotto di cicli disgiunti a meno dell'ordine dei fattori]


\end{thm}



















\newpage
\section{Algebra multilineare}

\subsection{Prodotto tensoriale}


\begin{defn}[Prodotto tensoriale]

\label{defn:prodotto tensoriale}
\end{defn}







\begin{defn}
DEFINISCI TRACCIA DEL PRODOTTO TENSORE, OVVERO 

\[ tr(f\otimes g)\]

\end{defn}


\begin{thm}
Se $f:V\to V$ e $g:W\to W$ sono endomorfismi di spazi vettoriali, allora vale la formula

\[tr(f\otimes g) = tr(f) tr(g)  \]

\end{thm}

\textsc{Dimostrazione:}


\subsection{Prodotto esterno e prodotto simmetrico}



\begin{defn}[Prodotto esterno]


\label{defn:prodotto esterno}
\end{defn}





\begin{thm}[Dimensione del prodotto esterno]



\label{thm:prodotto esterno}
\end{thm}








\begin{defn}[Prodotto simmetrico]


\label{defn:prodotto simmetrico}
\end{defn}





\begin{thm}[Dimensione del prodotto simmetrico]



\label{thm:prodotto simmetrico}
\end{thm}









\newpage
\section{Prime proprietà delle rappresentazioni}

\begin{defn}[Rappresentazione]

\end{defn}


\begin{defn}[Sottospazio invariante]

\end{defn}


\subsection{Operazioni con le rappresentazioni}

\begin{defn}[Somma di rappresentazioni]

\label{defn:somma di rappresentazioni}
\end{defn}

\begin{defn}[Prodotto di rappresentazioni]

\label{defn:prodotto di rappresentazioni}
\end{defn}



\begin{defn}[Rappresentazione regolare]

\label{defn:rappresentazione regolare}
\end{defn}

\begin{exmp}[La rappresentazione regolare di $S_3$]


\end{exmp}




\begin{thm}[Lemma di Schur]


\end{thm}








\newpage
\section{Teoria dei caratteri}


\begin{defn}
Sia $\rho: G \to V_\rho$ una rappresentazione di un gruppo $G$. Definiamo carattere di $\rho$ la funzione che associa ad ogni elemento del gruppo $G$ la traccia della matrice associata all'elemento, ovvero

\[\chi_\rho(s) := tr (\rho(s)) \qquad \forall s \in G \]
Notare che $\chi$ è una funzione che va dal gruppo in $\C^*$, ovvero $\chi: G \to \C^*$

\end{defn}

Vediamo delle proprietà elementari del carattere

\textsc{Osservazioni:}
\begin{enumerate}
	\item Se $dim(\rho) = 1$ allora il carattere di $s$ è uguale a $\rho(s)$
	\item $\chi_{\rho_1} = dim(\rho_1)$ \footnote{Al solito $\rho_1$ è la rappresentazione che manda ogni elemento nell'identità di $V_\rho$}
	\item $\chi_{\rho + \sigma}(s) = \chi_\rho(s) + \chi_\sigma(s)$.\\ 
	Questo è dovuto al fatto che la somma di rappresentazioni si può scrivere come matrice a blocchi. Una volta scritto così è evidente il risultato.
	\item $\chi_{\rho\sigma}(s) = \chi_\rho(s)\chi_\sigma(s)$.\\ 
	Questo deriva dal fatto che in generale se $f: V \to V$ e $g: W \to W$ sono endomorfismi di spazi vettoriali, allora vale $tr(f \otimes g) = tr(f)tr(g)$
	
	\item $\chi_{\rho^*}(s) = \overline{\chi_\rho(s)}$.\\
			Se abbiamo un gruppo finito \footnote{Ricordiamo che $\rho^*(s) = (\rho(s)^{-1})^*$}, allora $\exists n | (\rho(s))^n = id$, per cui tutti gli autovalori di $\rho(s)$ sono radici ennesime dell'unità e $\rho(s)$ è diagonalizzabile LINKA IL PUNTO IN CUI LO DIMOSTRI. Dato che possiamo scrivere $\rho(s)$ in una base in modo che sia diagonale per ogni $s$, è evidente che gli autovalori dell'inversa saranno l'inverso degli autovalori, ma dato che hanno modulo 1, l'inverso è uguale al coniugio. 
 \end{enumerate}



\end{document}
