\documentclass[11pt]{article}


\usepackage{etex}
\usepackage[T1]{fontenc}
\usepackage[utf8]{inputenc}
\usepackage[italian]{babel}
\usepackage[a4paper]{geometry}
\usepackage[pdftex]{graphicx}

\usepackage{amsmath}
\usepackage{amssymb}
\usepackage{amsthm}
\usepackage{booktabs}
\usepackage{paralist}
\usepackage{subfig}
\usepackage{array}
\usepackage{xy}
\usepackage{multicol}
%\usepackage{slashbox}
\usepackage{fancyhdr}
\usepackage{makeidx}
\usepackage{hyperref}
\usepackage{wrapfig}
\usepackage[T1,OT1]{fontenc} 
\usepackage[nohug,small]{diagrams}


\usepackage{grffile}
\usepackage{tikz}
\usepackage{pgf,tikz}
\usetikzlibrary{shapes.geometric,calc}

\usetikzlibrary{arrows}
\topmargin 0cm
\oddsidemargin 0cm
\evensidemargin 0cm
\textwidth 16.5cm
\textheight	23.5cm
\marginparwidth 2cm
\marginparpush 2cm



\title{Dispense del corso di Teoria della Rappresentazione}
\author{Fabio Zoratti}
\date{\today}



\makeindex

\theoremstyle{plain}
\newtheorem{thm}{Teorema}[section]
\newtheorem{lemma}[thm]{Lemma}
\newtheorem{prop}[thm]{Proposizione}
\newtheorem{post}[thm]{Postulato}
\newtheorem*{cor}{Corollario}

\theoremstyle{definition}
\newtheorem{defn}{Definizione}[section]
\newtheorem{exmp}{Esempio}[section]
\newtheorem{prob}{Problema}[section]
\newtheorem{hint}{Suggerimento}[section]
\newtheorem{sol}{Soluzione}[section]
\newtheorem*{rem}{Osservazione}

\theoremstyle{remark}
\newtheorem*{note}{Nota}





\newcommand{\C}{\mathbb{C}}
\newcommand{\dsum}{\displaystyle\sum}
\newcommand{\dint}{\displaystyle\int}






















\begin{document}
\maketitle





\newpage
\section{Teoria dei gruppi}

\begin{defn}[Gruppo] Un gruppo è un insieme con associata un operazione binaria $\cdot : G\times G \to G$ che gode di alcune proprietà
\begin{enumerate}
	\item Associatività \quad $(ab)c = a(bc)$
	\item Esistenza unità \quad $ea = ae = a$
	\item Esistenza inverso $a'$ per ogni elemento $a$ \quad $a' a = a a' = e$
\end{enumerate}

\end{defn}

\paragraph{Esempi}
\begin{enumerate}
	\item $\mathbb{Z}, \mathbb{Q}, \mathbb{R}, \mathbb{C}$ con l'operazione di somma.
	\item $\mathbb{Q}^*, \mathbb{R}^*, \mathbb{C}^*$ con l'operazione di moltiplicazione. (Senza lo 0)
	\item $GL_n(\mathbb{R})$ oppure $GL(V)$
	\item $f:I\to I $ biunivoca, con $I$ insieme e con l'operazione di composizione. Nel caso in cui $I$ sia un insieme finito, tanto vale scegliere $I = \{1,2,3,\ldots, n\}$. In tal caso questo gruppo si chiama $S_n$
\end{enumerate}

\paragraph{Alcuni teoremi elementari}
\begin{enumerate}
	\item L'unità $e$ è unica
	
	Dimostrazione: supponiamo per assurdo che siano due distinte, $e, e'$. Allora vale
	
	\[e = ee' = e' \qed\]
	
	\item L'inverso è unico.

	Dimostrazione:
	
	Supponiamo per assurdo che siano due, $a', a''$
	
	\[(a' a)a'' = a'(aa'') \Rightarrow e a'' = a' e \qed \]
	
	\item Se ho $a_1, a_2, \ldots, a_n$, il prodotto di questi termini è ben definito senza bisogno di parentesi
	\item Esistono le potenze, ovvero $\forall k \in \mathbb{Z}, \forall a \in G \exists b\in G | a^k = b$
	
	Vale sempre la regola
	\[a^{k+h} = a^k \cdot a^h \]
	
	Ricorda che 
	
	\[ (ab)^{-1} = b^{-1}a^{-1}\]

\end{enumerate}



\begin{defn}[Sottogruppo]
Sia $G$ un gruppo, $H\subseteq G$ si dice sottogruppo di $G$ se:
\begin{itemize}
	\item $e\in H$
	\item $x,y\in H \Rightarrow xy\in H$
\end{itemize}

\end{defn}

\begin{defn}[Sottogruppo normale]
Sia $G$ un gruppo, $H\subseteq G$ si dice \textit{normale} in $G$ se
\[
	\forall h\in H, \forall g\in G\qquad ghg^{-1}\in H
\]
\end{defn}

\begin{defn}[Gruppo quoziente]

\end{defn}

\begin{defn}[Classi di coniugio]


\end{defn}


\begin{exmp}[Le classi di coniugio di $GL_n(\C)$]

\end{exmp}


\begin{defn}[Prodotto di gruppi]

\end{defn}


\begin{defn}[Omomorfismo (isomorfismo) di gruppi]
Siano $G$ ed $H$ gruppi, un'applicazione $\varphi:G\to H$ si dice \textit{omomorfismo di gruppi} se
\[
	\forall g_1,g_2\in G\qquad \varphi(g_1 g_2)=\varphi(g_1)\varphi(g_2)
\]
dove la prima moltiplicazione è fatta in $G$ mentre la seconda in $H$.
Se $\varphi$ è bigettiva, allora si dice \textit{isomorfismo}.
\end{defn}


\begin{defn}[Azione di un gruppo su un insieme] Sia $G$ un gruppo e $I$ un insieme. Definiamo un azione $a$ di $G$ su $I$ una funzione $a:G\times I \to I$ che rispetti la regola di composizione, ovvero che se $h,h\in G$ e $i \in I$, valga

\[ a(h,a(g,i)) = a(hg, i) \]

Normalmente si usa una notazione abbreviata in cui invece di scrivere $a(g,i)$ si scrive direttamente $g\cdot i$ o addirittura $gi$


\label{defn:azione}
\end{defn}


\begin{defn}[Azione transitiva]



\label{defn:azione transitiva}
\end{defn}


\begin{defn}[Orbita di un elemento]


\label{defn:orbita}
\end{defn}




\begin{defn}[Azione semplicemente transitiva]

\end{defn}


\begin{defn}[Funzione $G$ equivariante]

Dato un gruppo $G$ che agisce su due insiemi $I$ e $J$, una funzione $\phi: I \to J$ si dice $G$ equivariante se 

\[ \phi(s \cdot_I i) = s \cdot_J \phi(i) \qquad \forall s \in G, \ \ \forall i \in I \]


\end{defn}


















\newpage
\subsection{Proprietà dei gruppi abeliani}



\begin{thm}Ogni gruppo abeliano finito è isomorfo al prodotto di gruppi ciclici.


\end{thm}

\begin{rem} Sia $G$ un gruppo abeliano. Allora 

\[ |G| = card(\{Hom(G \to \C^*) \})\]


Se invece $G$ non è abeliano allora nella formula precedente all'uguale va sostituito un $>$

\end{rem}


\subsection{Proprietà del gruppi simmetrici}

\begin{thm}[Ogni elemento $\sigma \in S_n$ si scrive in modo unico come prodotto di cicli disgiunti a meno dell'ordine dei fattori]


\end{thm}

\begin{prop}Il segno di un ciclo di lunghezza $k$ è esattamente $(-1)^{k-1}$


\end{prop}



\subsection{Proprietà dei gruppi ciclici}

\begin{rem}
Due gruppi ciclici dello stesso ordine sono isomorfi
\end{rem}

\subsection{Proprietà dei gruppi diedrali}









\newpage
\section{Algebra multilineare}
\subsection{Alcune generalizzazioni di algebra lineare}

\begin{defn}[Base di uno spazio vettoriale]

\end{defn}


\begin{lemma}
 Sia $e:I\to V$ una base di $V$ e $W$ uno spazio vettoriale. $f: I \to W$ una funzione. Allora $\exists! \phi: V \to W$ lineare tale che

\[\phi(e_i) = f_i \]

Inoltre $\phi$ è un isomorfismo $\Leftrightarrow$ $f$ è una base.
\end{lemma}


\subsection{Prodotto tensoriale}


% \[ \tridiag{V\times W}{ \otimes }{V\otimes W}{\phi}{Z}{f} \]
  


\begin{defn}[Prodotto tensoriale]

\label{defn:prodotto tensoriale}
\end{defn}


\begin{prop}
Se ho due prodotti tensoriali $V \otimes W$ e $V \overline{\otimes} W$, allora esiste un unico isomorfismo $\phi: V \otimes W \to V \overline{\otimes} W$ tale che

\[ \phi (v\otimes w) = v \overline{\otimes} w\]
\end{prop}


\begin{note}
\'E importante notare che non tutti gli elementi $z \in V \otimes W$ si scrivono come $z = v \otimes w$. In particolare, per fare un esempio concreto che mostra che questa cosa non funziona, prendiamo $W = V^*$. Vedremo fra poco che $V\otimes V^*$ è canonicamente isomorfo allo spazio delle applicazioni bilineari da $V$ in $\C$, che sappiamo scriverlo come matrici $n\times n$. Tuttavia se un elemento si scrive in termini di matrici come $z = v\otimes w$, allora la matrice associata a $z$ in una base avrà rango al massimo 1, ben lontano da coprire tutto lo spazio.
\end{note}


\begin{prop}
\[\langle\{ v \otimes w | v \in V, w \in W\} \rangle  = V \otimes W\]

\end{prop}


\begin{defn}[Prodotto tensoriale di mappe lineari]

\end{defn}

\begin{rem}

\[ id_V \otimes id_W = id_{V\otimes W}\]
\end{rem}




\begin{prop}

Se $e_i$ è una base di $V$ e $f_i$ è una base di $W$ allora $e_i \otimes f_j$ è una base di $V \otimes W$
\end{prop}


\begin{cor}
\[dim(V \otimes W) = dim V \cdot dim W \]

\end{cor}

















\begin{defn}
DEFINISCI TRACCIA DEL PRODOTTO TENSORE, OVVERO 

\[ tr(f\otimes g)\]

\end{defn}


\begin{thm}
Se $f:V\to V$ e $g:W\to W$ sono endomorfismi di spazi vettoriali, allora vale la formula

\[tr(f\otimes g) = tr(f) tr(g)  \]

\end{thm}

\textsc{Dimostrazione:}


\subsection{Prodotto esterno e prodotto simmetrico}



\begin{defn}[Prodotto esterno]


\label{defn:prodotto esterno}
\end{defn}





\begin{thm}[Dimensione del prodotto esterno]



\label{thm:prodotto esterno}
\end{thm}








\begin{defn}[Prodotto simmetrico]


\label{defn:prodotto simmetrico}
\end{defn}





\begin{thm}[Dimensione del prodotto simmetrico]



\label{thm:prodotto simmetrico}
\end{thm}
























\newpage
\section{Prime proprietà delle rappresentazioni}

\begin{defn}[Rappresentazione] Sia $G$ un gruppo. Una rappresentazione $\rho$ di $G$ è una coppia composta da uno spazio vettoriale di dimensione qualsiasi $V_\rho$ e una funzione $\rho: G \to GL(V_\rho)$ che manda ciascun elemento del gruppo in un'applicazione lineare di $V_\rho$, ovvero un suo endomorfismo. Affinché $\rho$ sia una rappresentazione deve essere un omomorfismo di gruppi, ovvero in parole semplici deve rispettare la regola di composizione. In formule, se $s, t \in G$ deve valere

\[ \rho(st) v = \rho(s)\rho(t) v \qquad \forall v \in V_\rho, \quad \forall s,t \in G\]

La dimensione di $V_\rho$ viene detta grado della rappresentazione.

\end{defn}

\begin{prop} $\rho(G)$ è evidentemente un sottogruppo di $GL(V_\rho)$, quindi esistono sempre inversi, potenze e tutte le cose che valgono per i gruppi.

\end{prop}


\textbf{Esempi.}
\begin{enumerate}
	\item La rappresentazione banale, di grado qualsiasi, indicata con $\rho_1$ che manda qualsiasi elemento di $g$ nell'identità di $V_\rho$, ovvero
	
	\[ \rho(s ) = id_{V_\rho} \qquad \forall s \in G\]
	\item Dato $S_n$, il segno di un elemento $s\in S_n$ è una rappresentazione di grado 1. Infatti si ha $sgn(st) = sgn(s) sgn(t)$.
	\item L'azione naturale di $S_n$ sui vettori della base. Prendiamo quindi $G = S_n$ e uno spazio vettoriale di dimensione $n$, che sarà sicuramente isomorfo a $\C^n$. Prendiamo la base canonica di $\C^n$ e la chiamiamo $e_i$. Descriviamo la rappresentazione $\rho: S_n \to GL(\C^n)$ dicendo cosa fa agli elementi della base. Per linearità si estenderà a tutto lo spazio.
	
	\[ \rho(s) e_i = e_{s(i)}\]
	
	Notare che in questo caso $deg(\rho) = n$. Notiamo inoltre che se rappresentiamo nella base canonica le matrici associate a $\rho(s)$ queste matrici sono unitarie. Inoltre, ogni colonna (e anche ogni riga) contiene esattamente un 1 e tutti gli altri sono 0.
	
	Prendiamo come esempio $S_3$ e vediamo cosa succede. Notiamo innanzitutto che $ |S_3| = 3! = 6$
	FINISCI DI SCRIVERE
\end{enumerate}








\begin{prop}
Sia $G$ un gruppo finito e $\rho: G \to GL(V_\rho)$ una sua rappresentazione. Allora $\forall g \in G$ la matrice $\rho(g)$ ammette una base di autovettori in $V_\rho$, ovvero è diagonalizzabile. Inoltre, tutti gli autovalori di $\rho(g)$ sono radici $n-$esime dell'unità.

\textsc{Nota bene:} Per ogni matrice in generale la base è diversa, quindi le varie matrici in generale \textbf{non} sono simultaneamente diagonalizzabili. In particolare, tutte le matrici $\rho(s)$ sono simultaneamente diagonalizzabili $\Leftrightarrow$ $G$ è abeliano. 

\label{prop:diagonalizzabilita rappresentazioni}
\end{prop}

\textsc{Dimostrazione:} Se $G$ è un gruppo finito, allora $\exists k | g^k = e$\footnote{Dato che $g$ è finito, se prendo l'insieme delle potenze $I = \{g^k| k\in \mathbb{Z}\}$, proprio perchè $G$ è finito si ha che $I$ ha un numero finito di elementi, quindi ci saranno $m,n \in \mathbb{Z}$ tali che $g^m = g^n = h$. Dato che nei gruppi esiste l'inverso, sarà $g^{n-m} = e$}. Dato che $\rho:G\to GL(V_\rho)$ mantiene queste proprietà in quanto omomorfismo, dovrà essere

\[ \rho(g)^k = id\]

Con il polinomio minimo si mostra facilmente che $\rho(g)$ è diagonalizzabile. MATEMATICI SCRIVETE IL PERCH\'E. Inoltre da questa formula è anche evidente che tutti gli autovalori di $\rho(g)$ hanno modulo $1$ e in particolare saranno radici $k-$esime dell'unità.

Ricordiamo un teorema di algebra lineare per finire l'ultima parte della dimostrazione: due endomorfismi di uno spazio vettoriale diagonalizzabili sono simultaneamente diagonalizzabili $\Leftrightarrow$ commutano. Da questo teorema segue facilmente la seconda parte dell'enunciato. \qed





\begin{defn}[Omomorfismo di rappresentazioni] 
%Siano $G, H$ due gruppi e $\rho: G \to V_\rho$ e $\sigma: H \to V_\sigma$ due loro rappresentazioni. Una funzione lineare da $V_\rho \to V_\sigma$ \footnote{Ovvero un omomorfismo da $V_\rho$ a $V_\sigma$} è un omomorfismo di rappresentazioni se rispetta la regola di composizione


%\[ \qquad \forall v,w \in V_\rho, V_\sigma\]


\end{defn}



\begin{defn}[Rappresentazioni isomorfe]


\end{defn}




\paragraph{Rappresentazioni di grado 1}

\begin{thm}[Le classi di isomorfismo delle rappresentazioni di grado 1 sono gli omomorfismi da $G$ in $\C^*$]


\end{thm}


\begin{exmp}[Rappresentazioni di grado 1 di $C_n$]

\end{exmp}


\begin{exmp}[Rappresentazioni di grado 1 di $S_3$]

\end{exmp}

\begin{exmp}[Rappresentazioni di grado 1 di $C_n \times C_n$]


(generalizzazione a prodotto di $C_{n_i}$)
\end{exmp}














\newpage
\subsection{Operazioni con le rappresentazioni}

\begin{defn}[Somma di rappresentazioni]

\label{defn:somma di rappresentazioni}
\end{defn}

Osservazioni:

\begin{enumerate}
\item $\rho + \sigma \cong \sigma + \rho$
\item $\rho + (\sigma + \tau) \cong (\rho + \sigma ) + \tau$
\item Esiste l'elemento neutro che è la rappresentazione di grado 0 ma non esiste l'inverso.

\end{enumerate}





\begin{defn}[Prodotto di rappresentazioni]

\label{defn:prodotto di rappresentazioni}
\end{defn}


Osservazioni:


\begin{enumerate}
\item $1\otimes \rho \cong \rho$
\item $\rho \otimes \sigma \cong \sigma \otimes \rho$
\item $0 \otimes \rho \cong 0$
\item $\rho \otimes (\sigma \otimes \tau) \cong (\rho \otimes \sigma)\otimes \tau$
\item $\rho \otimes (\sigma_1 + \sigma_2) \cong \rho \otimes \sigma_1 + \rho \otimes \sigma_2$

\end{enumerate}





\begin{defn}[Rappresentazione duale]

\label{defn:rappresentazione duale}
\end{defn}

Osservazione: vale

\[ (\rho + \sigma)^* \cong \rho^* + \sigma^* \]

E l'isomorfismo è canonico. SCRIVI DIMOSTRAZIONE.





\begin{defn}[Rappresentazione regolare]

\label{defn:rappresentazione regolare}
\end{defn}

\begin{exmp}[La rappresentazione regolare di $S_3$]


\end{exmp}


\begin{thm}

\[R_G \cong \dsum_i deg(\rho_i) \rho_i \]

\end{thm}



\subsection{Sottospazi invarianti e scomposizione delle rappresentazioni}


\begin{defn}[Sottospazio invariante]

\end{defn}

\begin{defn}[Sottorappresentazione]


\end{defn}



\begin{defn}[Rappresentazione irriducibile]

\end{defn}

\begin{rem} Normalmente la cosa che si fa più spesso in teoria della rappresentazione è cercare di scomporre la rappresentazione di un gruppo come somma di rappresentazioni irriducibili. Vedremo quindi adesso diversi teoremi che ci aiuteranno in questi problemi.

\end{rem}



\begin{exmp}[Rappresentazione regolare di $S_3$]


\end{exmp}



\begin{thm}[Le rappresentazioni di un gruppo finito sono completamente riducibili]

\end{thm}

\begin{prop}[Prodotto hermitiano invariante]

\end{prop}


\begin{lemma}
Sia $h: V_\rho \times V_\rho \to \C$ una forma hermitiana definita positiva e invariante per $\rho: G \to GL(V_\rho)$ e sia $\rho|_W: G \to GL(W)$ una sottorappresentazione di $\rho$. Allora se $W^\perp$ è l'ortogonale di $W$, $\rho|_{W^\perp}: G \to GL(W^\perp)$ è una sottorappresentazione.




\end{lemma}






\begin{lemma}
Sia $\rho: G \to GL(V_\rho)$ una rappresentazione di un gruppo finito $G$. Sia $\rho|_W: G \to GL(W)$ una sottorappresentazione di $\rho$. Allora esiste una sottorappresentazione $\sigma: G \to GL(W')$ tale che

\[\rho = \rho|_W + \sigma \]
\end{lemma}






\begin{rem} Notare che il teorema precedente è falso per gruppi finiti. (Esempio con $\mathbb{Z}^+$ che Salvatore non ha scritto con cura. Porco salvatore)



\end{rem}



\begin{thm} Siano $\rho: G \to GL(V_\rho)$ e $\sigma: G \to GL(V_\sigma)$ sono rappresentazioni di $G$ e $f: V_\rho \to V_\sigma$ è un omomorfismo di rappresentazioni, allora $Im(f)$ è una sottorappresentazione di $\sigma$ e $Ker(f)$ è una sottorappresentazione di $V_\rho$

\end{thm}





\begin{thm}Sia $G$ un gruppo abeliano finito. Allora ogni rappresentazione di $G$ è isomorfa alla somma di rappresentazioni di grado 1.



\end{thm}


\begin{prop} La rappresentazione regolare $\mathcal{R}$ di $C_n$ è isomorfa alla somma delle $n$ rappresentazioni irriducibili di grado 1 di $C_n$.

\end{prop}


\begin{lemma}
Date $\rho_1, \rho_2, \sigma$ rappresentazioni di $G$, allora

\[Hom(\rho_1 + \rho_2, \sigma) \cong Hom(\rho_1, \sigma) \oplus Hom(\rho_2, \sigma)\]

\end{lemma}


\begin{thm}[Lemma di Schur]


\end{thm}



\begin{thm}
Sia $\rho: G \to GL(V_\rho)$ una rappresentazione e 

\[\rho = \dsum_{i=1}^N n_i \rho_i \]

una sua scomposizione come somma di rappresentazioni irriducibili a due a due non isomorfe. Allora la scomposizione è unica.
\end{thm}



\begin{lemma}
Sia $\rho$ una rappresentazione di $G$ e $\mathcal{R}$ la sua rappresentazione regolare. Allora 
\[deg(\rho) = dim(Hom(\mathcal{R}, \rho))\]
\end{lemma}




\begin{thm}
Sia $\mathcal{R}$ la rappresentazione regolare di $G$, un gruppo finito, e sia 

\[ \mathcal{R} = \dsum_{i=1}^Nn_i \rho_i\]

Con $\rho_i$ irriducibili e a due a due non isomorfe. Allora ogni rappresentazione irriducibile di $G$ è isomorfa ad una $\rho_i$. Inoltre $n_i = deg(\rho_i)$ 
\end{thm}



\begin{cor}
Se $G$ è abeliano allora ha $|G|$ rappresentazioni irriducibili di grado 1 e $\mathcal{R}$ è la somma di queste.
\end{cor}



\begin{cor}
Sia $G$ un gruppo finito. $G$ ha un numero finito di rappresentazioni irriducibili, a meno di isomorfismi. Inoltre
\[|G| = \dsum n_i^2\]
\end{cor}



















\newpage
\section{Teoria dei caratteri}


\begin{defn}
Sia $\rho: G \to GL(V_\rho)$ una rappresentazione di un gruppo $G$. Definiamo carattere di $\rho$ la funzione che associa ad ogni elemento del gruppo $G$ la traccia della matrice associata all'elemento, ovvero

\[\chi_\rho(s) := tr (\rho(s)) \qquad \forall s \in G \]
Notare che $\chi$ è una funzione che va dal gruppo in $\C^*$, ovvero $\chi: G \to \C^*$

\end{defn}

Vediamo delle proprietà elementari del carattere

\textsc{Osservazioni:}
\begin{enumerate}
	\item Se $dim(\rho) = 1$ allora il carattere di $s$ è uguale a $\rho(s)$
	\item $\chi_{\rho_1} = dim(\rho_1)$ \footnote{Al solito $\rho_1$ è la rappresentazione che manda ogni elemento nell'identità di $V_\rho$}
	\item $\chi_{\rho + \sigma}(s) = \chi_\rho(s) + \chi_\sigma(s)$.\\ 
	Questo è dovuto al fatto che la somma di rappresentazioni si può scrivere come matrice a blocchi. Una volta scritto così è evidente il risultato.
	\item $\chi_{\rho\sigma}(s) = \chi_\rho(s)\chi_\sigma(s)$.\\ 
	Questo deriva dal fatto che in generale se $f: V \to V$ e $g: W \to W$ sono endomorfismi di spazi vettoriali, allora vale $tr(f \otimes g) = tr(f)tr(g)$
	
	\item $\chi_{\rho^*}(s) = \overline{\chi_\rho(s)}$.\\
			Se abbiamo un gruppo finito \footnote{Ricordiamo che $\rho^*(s) = (\rho(s)^{-1})^*$}, allora $\exists n | (\rho(s))^n = id$, per cui tutti gli autovalori di $\rho(s)$ sono radici ennesime dell'unità e $\rho(s)$ è diagonalizzabile\footnote{Si veda la proposizione \ref{prop:diagonalizzabilita rappresentazioni}}. Dato che possiamo scrivere $\rho(s)$ in una base in modo che sia diagonale per ogni $s$, è evidente che gli autovalori dell'inversa saranno l'inverso degli autovalori, ma dato che hanno modulo 1, l'inverso è uguale al coniugio. 
 \end{enumerate}





\subsection{Esempi di rappresentazioni di gruppi finiti}



























\end{document}
